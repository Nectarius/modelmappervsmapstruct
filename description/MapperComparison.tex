\documentclass{article}
% XeLaTeX
\usepackage{xltxtra}
\usepackage{xunicode}

% Fonts
\setmainfont{Liberation Mono}
\newfontfamily\cyrillicfont{Liberation Mono}
\setmonofont{Liberation Mono}

\setcounter{page}{1}

\setlength{\textheight}{21.6cm}

\setlength{\textwidth}{14cm}

\setlength{\oddsidemargin}{1cm}

\setlength{\evensidemargin}{1cm}

\usepackage{listings}
\usepackage{color}

\definecolor{dkgreen}{rgb}{0,0.6,0}
\definecolor{gray}{rgb}{0.5,0.5,0.5}
\definecolor{mauve}{rgb}{0.58,0,0.82}

\lstset{
  language=Java,
  aboveskip=3mm,
  belowskip=3mm,
  showstringspaces=false,
  columns=flexible,
  basicstyle={\small\ttfamily},
  numbers=none,
  numberstyle=\tiny\color{gray},
  keywordstyle=\color{blue},
  commentstyle=\color{dkgreen},
  stringstyle=\color{mauve},
  breaklines=false,
  breakatwhitespace=false,
  tabsize=3
}

% Lang
\usepackage{polyglossia}
\setmainlanguage{russian}
\setotherlanguage{english}

\date{}

\begin{document}

\centerline{\Large{\bf Сравнение ModelMapper и MapStruct и Orika на примерах }}

\section{Мэппинг нескольких полей на единственное}


 \begin{itemize}
 \item Domain :
 \begin{lstlisting} 
  class  Note {
     private String title;
     private String brief;
     private String text;
     private String type;
  }
  \end{lstlisting}
 \item ViewObject
 \begin{lstlisting} 
 class SimpleNoteView {
      private String description;
     private String type;
  }
  \end{lstlisting}
 \end{itemize}
  \textbf{@Test} twoInOne()
 
 \paragraph{Задача:}
 Мы хотим маппить из Note поля title и brief на description
 
  \paragraph{MapStruct:}
  Мэппинг удалось решить с помощью expression а.  
  
\begin{lstlisting} 
 @Mapper(componentModel = "spring")
 public interface NoteMapper { 
     @Mapping(target = "description",
             expression = "java(  String.valueOf(note.getTitle() + ' ' + note.getBrief()) )")
     NoteView noteToNoteView(Note note);
    \end{lstlisting}
    //Фрагмент сгенерированного кода:
  \begin{lstlisting}   
     NoteView noteView = new NoteView(); 
     noteView.setType( note.getType() );
     noteView.setDescription( String.valueOf(note.getTitle() + ' ' + note.getBrief()) );
 
  \end{lstlisting}
  \paragraph{Замечание: }
    В перспективе возможно для этого можно попробовать использовать decorator. 
    К сожалению в данный момент он не поддерживается для Spring конфигурации
    
    
     \paragraph{ModelMapper:}
     В Model Mapper есть возможность делать мэппинг для полей с помощью  PropertyMap.
     Но возникла проблема из за proxies types
     \begin{lstlisting} 
     public class NoteMap extends PropertyMap<Note, NoteView> {
             protected void configure() {
                 String description = source.getTitle() + ' ' + source.getBrief();
                 map().setDescription(description);
             }
     }
     \end{lstlisting}
     //Будет брошено исключение 
     "Cannot map final type java.lang.StringBuilder."
      Удалось решить только конвертером.
      \begin{lstlisting} 
          private Converter<Note, NoteView> noteConverter = new AbstractConverter<Note, NoteView>() {
                  protected NoteView convert(Note note) {
                      NoteView noteView = new NoteView();
                      noteView.setDescription(note.getTitle() + ' ' + note.getBrief());
                      return noteView;
                  }
              };
         modelMapper.addConverter(noteConverter);
     \end{lstlisting}
\section{Мэппинг с условием} 
    
    \paragraph{Задача:}
Допустим необходимо смаппить на commentary title если type == common 
Добавим в NoteView поле commentary

\paragraph{ModelMapper:}
В ModelMapper есть Condition.
Таким образом будет выглядеть условие
\begin{lstlisting} 
      Condition<Note, NoteView> typeIsCommon = new Condition<Note, NoteView>() {
             @Override
             public boolean applies(MappingContext<Note, NoteView> context) {
                 System.out.println("test");
                 Note note = (Note)context.getParent().getSource();
                 boolean flag = note.getType().equals("common");
                 return flag;
             }
         };
\end{lstlisting}
Добавим в PropertyMap     
     \begin{lstlisting} 
           public class NoteMap extends PropertyMap<Note, NoteView> {
                   protected void configure() {
                       when(typeIsCommon).map().setCommentary(source.getTitle());
                   }
               }
     \end{lstlisting}
и в ModelMapper 
\begin{lstlisting} 
           modelMapper.addMappings(new NoteMap());
     \end{lstlisting}
     
  \paragraph{MapStruct:}
  
     Здесь приведу как вариант Статический метод + expression.  Дальше будет другой пример по аналогии с которым можно вместо статического метода
     использовать custom mapper.
     
     статический метод:
   \begin{lstlisting} 
             public static String getCommentaryByType(Note note){
                       if(note.getType().equals("common")){
                           return note.getTitle();
                       }
                       return "";
                 };
        \end{lstlisting}  
     И Expression :   
   \begin{lstlisting} 
                     @Mapping(target = "commentary",
                                 expression = "java(  NoteConditions.getCommentaryByType(note) )")
                         NoteView noteToNoteVieww(Note note);
   \end{lstlisting}  
   
\section{Дженерики} 
 \paragraph{Задача:}
  Допустим есть AbstractNote с методом возвращающим абстрактный объект AbstractComment
  А нам нужно сделать маппинг его наследника с переопределённым методом.
  
  \paragraph{MapStruct:}
   \begin{lstlisting} 
  @Mappings({
              @Mapping(target = "commentary",
                      expression = "java( ((VelvetNote)note).getCommentary().toString()  )"),
              @Mapping(target = "advanceTitle",
                      expression = "java( ((VelvetNote)note).getAdvanceTitle()  )")
      })
      \end{lstlisting} 
  \paragraph{ModelMapper:}
  
  Подхватил свойства наследника без дополнительных настроек
  
  \section{Мэппинг сущностей тип полей которых может меняться от внешнего условия}
  \begin{itemize}
   \item Domain :
   \begin{lstlisting} 
   public class Control {  
       private Long id;   
       private String name;  
       private Details details;   
   
   public class Details {
       private String name;
       private String description;
   
    \end{lstlisting}
   \item ViewObject
   \begin{lstlisting} 
  public class ControlView {
      private String name;
      private String status;
      private DetailsView detailsView;
  \end{lstlisting}
  А DetailsView типом в зависимости от внешнего условия
  \begin{lstlisting} 
  public class AdminDetailsView extends DetailsView {
      private String description;
  или
  
  public class DetailsView {
      private String name;
    \end{lstlisting}
   \end{itemize}
   
   Удалось сделать решение без использования дополнительного конвертера только на Map Struct.
   
   \textbf{Основной Mapper с методом:} 
    \begin{lstlisting} 
    @Mapping(target = "detailsView", expression = "java(abstractDetailsMapper.fromDetailsView(control.getDetails(), isEmployee))")
        ControlView controlToControlView(Control control, Boolean isEmployee);
     \end{lstlisting}
    \textbf{Дополнительный :} 
     \begin{lstlisting} 
        @Mapper(componentModel = "spring")
        public interface DetailsMapper {
            DetailsView detailsToDetailsView(Details details);
            AdminDetailsView adminDetailsToDetailsView(Details details, Boolean isEmployee);
        }
         \end{lstlisting}
         isEmployee и есть параметр внешнее условие
         и custom mapper который делает выбор по условию
         \begin{lstlisting} 
        @Component
        public class AbstractDetailsMapper {
        @Autowired
        private DetailsMapper detailsMapper;
        public DetailsView fromDetailsView(Details details) {
            return detailsMapper.detailsToDetailsView(details);
        }
        public DetailsView fromDetailsView(Details details, Boolean isEmployee) {
            if (isEmployee) {
                return detailsMapper.adminDetailsToDetailsView(details, isEmployee);
            } else {
                 return detailsMapper.detailsToDetailsView(details);
            }               
         }
         \end{lstlisting}
  \section{О проверка маппинга} 
  
   \paragraph{ModelMapper:}      
         В ModelMapper можно проверить что все поля были смапплены с помощью вызова метода validate у mapper а.       
         Если мы в PropertyMap  закомментарим
         \begin{lstlisting}     
           map().setDescription(source.getText()); 
          \end{lstlisting}        
         То на проверки будет брошено исключение    
          \begin{lstlisting}      
         1) Unmapped destination properties found in TypeMap[Note -> NoteView]:
          \end{lstlisting}
    \paragraph{MapStruct:}   я не нашёл способов чтобы он бросал исключение при возникновении ошибок тем не менее, но он может выводить информацию в лог о полях который не были замапплены. 
\end{document}